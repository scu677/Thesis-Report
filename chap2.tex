\chapter{Design and Specifications}
\label{chap:figtab}
\label{chap}

\section{Specifications}

 The objective of this software is to design a dynamic user interface and a functional database system. With regard to that, the specification is made up of necessary features and functionality that appear on the user interface. The following are the major specification of this software:
\begin{description}
\item[$\bullet$] User Log in
\item[$\bullet$] Data Inventory
\item[$\bullet$] Searching Inventory(SI)
\item[$\bullet$] Make Reservations
\item[$\bullet$] Cancel Reservations
\item[$\bullet$] Add data
\item[$\bullet$] Update data
\end{description}
Registered users have access to the system with unique user name, so the user interface provide a log in form on the web page where user can submit their details to be stored in the database. The user interface is able to list the inventory of data stored in the database in difference categories. This is one of the main features on the user interface because it presents the content of the database. As there are bunch of data inventory on the user interface, it is necessary to have a function that allow users to search specify information from the inventory. Therefore the SI (search inventory) is a function designed to handle this issue. Another specification of this software is the ability to make reservations: this means that users are able to choose a machine from the inventory and reserve it for a fixed date and time. This function on the high level is meant to allow user to colonize a specific machine for running a particular task on that machine without interruption. When a reservation made is expired, another user will have the liberty to cancel that particular reservation to make the machine free. As new machines are added to the cluster, the Add-data function allow user to add details of machine to the database and update the inventory. Likewise, when user want to change a certain properties of the machines such as, memory size or disk, the Update-data function helps to perform this action.


\section{Design}
The design approach is based on the requirement to solve the issue of user interface, and system management. So we have designed this software with the intention to providing functional solution to the specifications. The design structure comprises the various  segments and element of the user interface (front end) combine with the database (back end)  and the control protocols. 
We designed a web interface that provides  flexible and dynamic access to the system with interactive functionality. This web page contains inputs and out put features that allow data to go in and out. The reason for choosing a web page instead of other options is to proving online dynamic and simultaneous  access to the system rather than performing manual command prompts. Another reason is to have a functional real time management capability in the Clowder. The web page is managed by a web server, which serves a channel between the user and the database. We designed a operational database management system that address the issue of system management. The database stores data from the web page and also retrieve data to the inventory on the web page via the server. As user log on to the their account on the web page the server will pass this information to the  database which is controlled by the main program (command protocol). The command protocols is responsible for handling the user interface activities and executing the database queries .  Below is a figure that describe the main structure of the design in terms of operation.
\pagebreak
\begin{figure}
\includegraphics[width = \linewidth]{Design}
\label{fig:Design structure}
\caption{The design structure}
\end{figure}


\subsection{User Interface}
The web interface represents the user interface (UI) where users get access to the system dynamically. It is the main part of the design because it represents the front end of the software.The web interface elements controls the basic functionality for data input and out put. It represent each feature as mentioned in the specification. This web interface structure is designed using  HTML template as the front end and GO programming language as the back end. It contains all the necessary elements required to have dynamic and flexible access to the system. These elements includes:
\begin{description}
  \item[$\bullet$] Forms
  \item[$\bullet$] Input controls
  \item[$\bullet$] Navigational component
  \item[$\bullet$] Containers
\end{description}
\begin{figure}[ht]
\includegraphics[width = \linewidth]{Design.eps}
\label{fig:Web Inteerface} 
\caption{User Interface Description}
\end{figure}


\subsection*{Forms}
 Forms are one the elements that provides the ability to input data to the database.The data is submitted through a HTML post request. The post forms are used for submitting data such as machines and reservations details. When this data is submitted with HTML form, the web server call the function handler to parse this data to the database. In this system design, the forms are used for creating new machines and reservation. The form has a text field for placing texts and submission button for initiating post request. 

\subsection*{Input Controls}
The input control in this web interface provides the necessary components that supports the functionality of the software. There is array of HTML input control, but we have chosen a few that satisfy our specifications. These input controls includes text field, buttons, date fields, drop down list and list box. They are used for both input and output functions like listing the inventory, inserting texts, viewing selection options and searching inventory. These controls are designed with HTML tags and adopted into Go functions with HTML template.

\subsection{Server}
The server is responsible for parsing all request from the web interface to the database and command protocol. When request is made on the web interface (for instance available machine on database), the server calls the  specific function of command protocol to handle this request by providing the necessary resources. Likewise the server interchange communication between the user interface and the database system for executions and requests. 

\begin{figure}[ht]
\includegraphics[width = \linewidth]{Design.eps}
\label{fig:Web Inteerface} 
\caption{Server description}
\end{figure}

\subsection{Database}
The database is another major component of this system, because it is provides storage and resource for the system. The data includes reservation records, machine details, disks, and NICs information. The database is designed with SQLite model using Go programming language as the back end tech. SQlite was used here because its a fast open source SQL engine. We created a functional database system that suites the specification of the software. 
The database scheme contains tables of machines, user record, reservation, NICs (network interface cards) and disks. Each table is designed to have a primary key (PK) for unique identification. One of the function of the software is the ability to make reservations.  So the reservation table has a foreign key (FK) reference to machine Id and user Id. This FK is used in reservation table to create a relationship between the reservation and  users or machines tables. For example, when a User make a reservation of a particular machine, the program takes the Users' id (FK) and machine Id to create a reservation record in the database. Below show a typical model of the database design.

 \begin{figure}[ht]
\includegraphics[width = \linewidth]{Design.eps}
\label{fig:Web Inteerface} 
\caption{Database design}
\end{figure}

\subsection{Command protocols}

\subsection{Design Data Flow}



